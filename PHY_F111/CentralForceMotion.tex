\documentclass[12pt]{article}

\usepackage{amsmath}
\usepackage{amssymb}
\usepackage{amsthm}
\usepackage{physics}

\pdfpkresolution=300

\theoremstyle{definition}
\newtheorem*{example}{Example}
\theoremstyle{definition}
\newtheorem*{defn}{Definition}

\begin{document}
\title{Central Force Motion}
\author{Sabyasachi Bhoi}
\date{ Dec 30, 2020 }
\maketitle

\textbf{Assumption:} All particles are rigid bodies.

\section{Degrees of Freedom}

\begin{center}
	\begin{tabular}{|c|c|c|c|}
		\hline
		No. of Particles & DoF & Constraints & Independent DoF \\
		\hline
		1 & 3 & 0 & 3 \\
		\hline
		2 & 6 & 1 & 5 \\
		\hline
		3 & 9 & 3 & 6 \\
		\hline
	\end{tabular}
\end{center}

Therefore, for $N$ particles, the Independent Degree of Freedom is defined as:
\[
	3N - \frac{N(N-1)}{2}
\]

\begin{center}
	\begin{tabular}{|c|c|}
		\hline 
		No. of Particles & Independent DoF \\
		\hline 
		4 & 6 \\
		\hline 
		5 & 5 \\
		\hline 
		6 & 3 \\
		\hline 
		7 & 0 \\
		\hline 
		8 & -4 \\
		\hline 
	\end{tabular}
\end{center}

Here, the Independent Degree of Freedom for 8 particle system doesn't make sense. The extra values constraints are considered \textit{redundant} constraints. 

\begin{itemize}
	\item If the number of constraints is equal to the degrees of freedom, then the system is stationary.
	\item If the number of constraints is greater than the degrees of freedom, then the system is unstable and would dissociate. 
\end{itemize}

\section{Translational invariance, Rotational invariance and Constants of motion}

\begin{align*}
	\vec{F} &=  -\frac{G m_1 m_2}{ \abs{\va{r}_1 - \va{r}_2}^3 } \va{r} \\
	U \qty(\va{r}_1, \va{r}_2) &=  - \frac{G m_1 m_2}{\abs{\va{r}_1 - \va{r}_2}}
\end{align*}

\end{document}




















