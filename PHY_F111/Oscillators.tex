\documentclass[12pt]{article}

\pdfpkresolution=600
\usepackage{amsmath}
\usepackage{amssymb}

\begin{document}

\title{Harmonic Oscillations}
\author{Sabyasachi Bhoi}
\date{Jan 24, 2021}
\maketitle

\section{Simple Harmonic Oscillators}
The general equation for a simple harmonic oscillator is of the form

\begin{equation*}
	\ddot{x} + \omega_0^2x = 0
\end{equation*}
Here, $\omega_0 = \sqrt{ \frac{k}{m} }$

\section{Damped Harmonic Oscillators}

The general equation for a damped harmonic oscillator is of the form

\begin{equation*}
	\ddot{x} + \gamma \dot{x} + \omega_0 x = 0
\end{equation*}
Here, $\gamma = \frac{b}{m}$ and $\omega = \sqrt{\frac{k}{m}}$.

We assume the general solution for this to be of the form
\begin{equation*}
	x = X(T) e^{\alpha t}
\end{equation*}
Here, $X(T)$ \textbf{cannot} be an exponential function.  
From this, we get the following equation

\begin{equation*}
	\ddot{X}(T) + \dot{X}(T) (2\alpha + \gamma) + X(T) (\alpha^2 - \alpha\gamma + \omega_0^2) = 0
\end{equation*}

After finding the eigenvalues for the equation $Dx = \lambda x$, we get
\begin{equation*}
	\lambda = \frac{-\gamma}{2} \pm \frac{\sqrt{\gamma^2 - 4\omega_0^2}}{2}
\end{equation*}

This gives rise to three different cases:
\subsection{ $\gamma^2 - 4 \omega_0^2 = 0$ }
This is the condition for \textbf{Critical Damping}.
Here, $\lambda = - \frac{\gamma}{2}$

\begin{equation*}
	x = (A + Bt) e^{-\gamma t}
\end{equation*}

\subsection{ $\gamma^2 -4\omega_0^2 > 0$ }
This is the condition for \textbf{Over damping} 
Here, $\lambda_1 = - \frac{\gamma}{2} - \frac{\gamma}{2} \sqrt{1 - \frac{4 \omega_0^2}{\gamma^2}} $ and $\lambda_2 = - \frac{\gamma}{2} + \frac{\gamma}{2} \sqrt{1 - \frac{4 \omega_0^2}{\gamma^2}} $

\begin{equation*}
	x = A e^{\lambda_1 t} + B e^{\lambda_2 t}
\end{equation*}

\subsection{ $\lambda^2 - 4\omega_0^2 < 0$ }
This is the case for \textbf{light damping}. This is the only case where the particle shows some form of oscillations. \\

Let $\omega_1 = \sqrt{ \frac{\lambda^2}{4} - \omega_0^2 } $

\begin{equation*}
	\therefore \lambda = \frac{-\gamma}{2} \pm i \omega_1
\end{equation*}

\begin{align*}
	\implies x &=  e^{- \frac{\gamma t}{2}} \left( A e^{i\omega_1} + B e^{i\omega_2} \right) \\
	\implies x &=  A e^{- \frac{\gamma t}{2}} cos \left( \omega_1 t + \phi \right)
\end{align*}

\subsubsection{Energy for lightly damped case}
\begin{equation*}
	E = E_0 e^{-\gamma t}
\end{equation*}

\section{Forced Oscillators}

\subsection{Undamped case}
\begin{align*}
	F(t) &=  F_0 cos\omega t \\
	m\ddot{x} + kx &=  F_0 cos\omega t \\
\end{align*}
\begin{equation}
	\label{eqn:FHM}
	\implies \ddot{x} + \omega_0^2x = \frac{F_0}{m} cos \omega t
\end{equation}

Here, we have to take into account two different frequencies:
$\omega_0$(natural frequency) and $\omega$(forced frequency).

Let the solution to this differential equation be $x = A cos {\omega t}$

\begin{equation*}
	\therefore \ddot{x} = -A \omega^2 cos \omega t
\end{equation*}
Putting this in equation \ref{eqn:FHM}, we get
\begin{equation*}
	A = \frac{F_0}{m \left( \omega_0^2 - \omega^2 \right)}
\end{equation*}

\begin{equation*}
	\therefore x_p = \frac{F_0}{m} \frac{1}{ \left( \omega_0^2 - \omega^2 \right) } cos \omega t
\end{equation*}
This is a \textbf{particular solution} to the equation.

\begin{equation*}
	x_h = B cos \left( \omega_0 t + \phi \right)
\end{equation*}
This is the equation for an undamped, natural oscillation. \\ \\
Therefore, the general equation for a \textbf{undamped forced oscillator} is
\begin{align*}
	\implies x &=  x_p + x_h \\
	x &=  \frac{F_0}{m} \frac{1}{ \left( \omega_0^2 - \omega^2 \right) } cos \omega t + B cos \left( \omega_0 t + \phi \right)
\end{align*}

\subsection{Damped case}

\begin{equation*}
	\ddot{x} + \gamma \dot{x} + \omega_0^2 x = \frac{F_0}{m} cos \omega t = f cos \omega t
\end{equation*}

To solve this, let's consider a \textit{companion equation} of the form
\begin{equation*}
	\ddot{y} + \gamma \dot{y} + \omega_0^2 y = \frac{F_0}{m} sin \omega t = f sin \omega t
\end{equation*}

Let $z(t) = x(t) + i y(t)$.

\begin{equation*}
	\therefore \ddot{z} + \gamma \dot{z} + \omega_0^2 z = f e^{i\omega t}
\end{equation*}

On assuming the solution for this differential equation to be $z = z_0 e^{i\omega t}$, We get 
\begin{equation*}
	A = \frac{F_0}{m} \frac{1}{\sqrt{ \left( \omega_0^2 - \omega^2 \right)^2 + \gamma^2\omega^2 }},
\end{equation*}
\begin{equation*}
	\phi = tan^{-1} \left( \frac{\gamma\omega}{\omega^2 - \omega_0^2} \right)
\end{equation*}

The particular solution is
\begin{equation*}
	z = z_0 e^{i\omega t} = A e^{i\phi} e^{i\omega t}
\end{equation*}

The real part for this is
\begin{equation*}
	x_p(t) = A cos \left( \omega t + \phi \right)
\end{equation*}

Solution for a naturally damped oscillation is:
\begin{equation*}
	x_h(t) = C_1 e^{\alpha_1 t} + C_2 e^{\alpha_2 t}
\end{equation*}

Therefore, the solution for a \textbf{Damped Forced Oscillator} is:
\begin{equation*}
	x(t) = A cos \left( \omega t + \phi \right) + C_1 e^{\alpha_1 t} + C_2 e^{\alpha_2 t}
\end{equation*}

\end{document}
