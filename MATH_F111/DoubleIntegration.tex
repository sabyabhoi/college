\documentclass[12pt]{article}

\usepackage{amsmath}
\usepackage{esint}
\usepackage{amssymb}
\usepackage{amsthm}
\usepackage{physics}
\pdfpkresolution=300

\theoremstyle{definition}
\newtheorem*{example}{Example}
\theoremstyle{definition}
\newtheorem*{defn}{Definition}

\begin{document}
\title{Double Integration}
\author{Sabyasachi Bhoi}
\date{ Dec 30, 2020 }
\maketitle

\section{Introduction}

\subsection{Single Integral}

\[
	\int_{a }^{b} {f(x)} \: d{x} 
\]

\[
	P = \left\{ x_0 = a, x_1, x_2, \dots x_k, x_{k+1}, \dots x_n = b \right\}
\]
$P$ is called a partition of the interval $ \left[ a, b \right]$

\subsubsection{Reimann Sum}
\[
	S_n = \sum^{n-1}_{k = 1} f(C_k) \Delta x_k, \quad \textrm{where}\; C_k \in \left[ x_k, x_{k+1} \right] 
\]

If $ \lim_{n \to \infty} S_n $ exists, then we say that 
\[
	\lim_{n \to \infty} S_n = \int_{a}^{b} {f(x)} \: d{x} 
\]

When $n \to \infty$, $\norm{P}$ is defined as the norm of partition.

\[
	\norm{P} = \max_{0 \leq i \leq n} \Delta x_k
\]

\section{Double Integration}

$f(x, y)$ defined over a rectangle $R$ where the $ a \leq x \leq b$, $c \leq y \leq d$. Then the double integration of this function in the given range will give the volume contained by the function.

Consider a small rectangle of length $\Delta x_k$ and width $\Delta y_k$.
\[
	\textrm{The area confined by this rectangle} =  \Delta A_k = \Delta x_k \times \Delta y_k
\]

Let $\left( c_k, d_K \right)$ be a point in the smaller rectangle.

\[
	S_n = \sum^{n-1}_{k = 0} f(c_k, d_k) \Delta A_k
\]
If $ \lim_{n \to \infty} S_n$ exists, then we say 

\[
	\lim_{n \to \infty} S_n = \iint\limits_R f(x, y) \,dA
\]

\subsection{Fubini's Theorem (first form)}
If $f(x, y)$ is continuous on $R$, where $R: a \leq x \leq b$ and $c \leq y \leq d$, then
\[
	\int_a^b \int_c^d f(x, y) \,dy\,dx = \int_c^d \int_a^b f(x, y) \,dx\,dy 
\]

\begin{example}
	Find $c$ such that 
	\[
		\int_0^1 \int_0^c \qty(2x + y) \, dx \, dy = 3
	\]

	\begin{align*}
		&\int_0^1 \qty[x^2 + xy]_0^c \, dy = \int_0^1 \qty[c^2 + cy] \, dy \\
		\implies & c^2 + \frac{c}{2} = 3 \\
		\implies & 2c^2 + c - 6 = 0 \\
		\implies & c = \frac{3}{2}, \; 2
	\end{align*}
\end{example}



\end{document}




















