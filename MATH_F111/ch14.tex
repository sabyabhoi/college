\documentclass[12pt]{article}

\usepackage{amsmath}
\usepackage{amssymb}
\usepackage{amsthm}

\theoremstyle{definition}
\newtheorem*{example}{Example}
\theoremstyle{definition}
\newtheorem*{defn}{Definition}

\begin{document}

\title{Partial Derivatives}
\author{Sabyasachi Bhoi}
\date{Dec 18, 2020}
\maketitle

\begin{defn}
	Suppose $D$ is a set of n-tuples of real numbers $(x_1, x_2, \cdots x_n)$. A \textbf{real-valued function} $f$ on $D$ is a rule that assigns a unique (single) real number
	\begin{equation*}
		w = f(x_1, x_2, \cdots, x_n)
	\end{equation*}
to each element in $D$. The set $D$ is the function's domain. The set of w-values taken on by $f$ is the function's range. 
\end{defn}

\begin{example}
	\begin{equation*}
		f(x, y) = \frac{sin xy}{x^2 + y^2 - 25}
	\end{equation*}
	Here, The domain is $D = \left\{ (x, y) \in \mathbb{R}^2 | x^2 + y^2 \neq 25 \right\}$.
\end{example}

\begin{example}
	\begin{equation*}
		f(x, y) = \sqrt{y - x^2}
	\end{equation*}
	Here, the domain is $D = \left\{ (x, y) \in \mathbb{R}^2 | y \geq x^2  \right\} $
\end{example}

\begin{defn}
	\textbf{Interior Point:} $(x_0, y_0) \in R$ is called an interior point if $\exists$ a disc centered at $(x_0, y_0)$ and contained in $R$ 
\end{defn}

\begin{example}
	$R = \left\{ (x, y) \mid x^2 + y^2 < 1 \right\}$. Here, $(0, 1)$ is \textbf{not} an interior point, because we cannot draw a disc of any radius greater than 0 which would be contained in this region.
\end{example}

\begin{defn}
	\textbf{Boundary Point:} $(x_0, y_0) \in \mathbb{R}$ if every disc centered at $(x_0, y_0)$ contains points inside and outside of $R$
\end{defn}

\begin{example}
	$R = \left\{ (x, y) \in \mathbb{R}^2 \mid x^2 + y^2 < 1 \right\}$. Here, boundary points of $R = \left\{ (x, y) \in \mathbb{R}^2 \mid x^2 + y^2 = 1 \right\}$
\end{example}

\begin{defn}
	\textbf{Open Set:} $R$ is called an open set if each point of $R$ is an interior point.
\end{defn}

\begin{defn}
	\textbf{Closed Set:} $R$ is called a closed set if it contains all of its boundary points.
\end{defn}

\begin{defn}
	\textbf{Level curve:} The set of points in the plane where a function $f(x, y)$ has a constant value $c$.
\end{defn}

\begin{example}
	\begin{equation*}
		f(x, y) = x + y - 1
	\end{equation*} 
	Level curve for $c = -1$ is 

	\begin{align*}
		&\left\{ (x_0, y_0) \in \mathbb{R}^2 \mid x_0 + y_0 -1 = -1 \right\}\\
		= & \left\{ (x_0, y_0) \in \mathbb{R}^2 \mid x_0 + y_0 = 0 \right\}
	\end{align*}
\end{example}

\begin{example}
	\begin{equation*}
		f(x, y) = x^2 + y^2
	\end{equation*}
	Level curve for $c = 0$:
	\begin{align*}
		&x^2 + y^2 = 0 \\
		\implies &x = y = 0
	\end{align*}

	Level curve for $c = 1$:
	\begin{equation*}
		x^2 + y^2 = 1
	\end{equation*}
\end{example}

\begin{defn}
	\textbf{Level Surface:} The set of points in space where the function $f(x, y, z)$ has a constant value $c$.
\end{defn}

\begin{example}
	\begin{equation*}
		f(x, y, z) = x^2 + y^2 + z^2
	\end{equation*}

	Level surface for $c = 1$:
	\begin{equation*}
		x^2 + y^2 + z^2 = 1 \quad (sphere)
	\end{equation*}
\end{example}

\section{Limit of a multivariable function}

\subsection{limit in one dimension}
\begin{defn}
	\begin{equation*}
		\lim_{x \to x_0} f(x) = L 
	\end{equation*}
	$\forall\;\varepsilon > 0$, $\exists\;\delta > 0$ such that
	\begin{equation*}
		\left| f(x) - L  \right| < \varepsilon \quad whenever \quad 0 < \left| x - x_0 \right| < \delta
	\end{equation*}
\end{defn}

\begin{example}
	\begin{equation*}
		\lim_{x \to 2} 5x-4 = 6
	\end{equation*}

	\textbf{Proof:} 
	\begin{equation*}
		\left| f(x) - L \right| < \varepsilon
	\end{equation*}

	\begin{equation*}
		L = 6 \quad x_0 = 2
	\end{equation*}

	\begin{align*}
		\left| 5x - 4 - 6 \right| &= \left| 5x -10 \right| \\
		&= 5 \left| x -2 \right| < \varepsilon
	\end{align*}

	Take $\delta = \varepsilon/5$

\end{example}

\subsection{limit in 2 dimensions}

\begin{defn}
	\begin{equation*}
		\lim_{(x, y) \to (x_0, y_0)} f(x, y) = L
	\end{equation*}

	If $\forall\;\varepsilon > 0$, $\exists\;\delta > 0$ such that
	\begin{equation*}
		\left| f(x, y) - L \right| < \varepsilon \quad whenever \quad 0 < \sqrt{(x - x_0)^2 + (y - y_0)^2} < \delta
	\end{equation*}
\end{defn}

\subsection{Properties of limits in 2D}

let $ \lim_{(x, y) \to (x_0, y_0)} f(x, y) = L$, $ \lim_{(x, y) \to (x_0, y_0)} g(x, y) = M$. Then

\begin{enumerate}
	\item $ \lim_{(x, y) \to (x_0, y_0)}  (f \pm g) = L \pm M$
	\item $ \lim_{(x, y) \to (x_0, y_0)}  (f \cdot g) = L \cdot M$
	\item If $M\neq0$, $ \lim_{(x, y) \to (x_0, y_0)}  \left( \frac{f}{g} \right) = \frac{L}{M}$
	\item $ \lim_{(x, y) \to (x_0, y_0)} f^n = L^n$
\end{enumerate}

\begin{example}
	\begin{equation*}
		\lim_{(x, y) \to (0, 0)} \frac{4xy^2}{x^2 + y^2}
	\end{equation*}

	To find this limit, we approach the point $(0, 0)$ through $y = mx$. 
	\begin{align*}
		&y = mx \\
		&= \lim_{x \to 0} \frac{4x \left( mx \right)^2}{x^2 + m^2x^2} \\ 
		&= \lim_{x \to 0}\;x \left( \frac{4m^2}{1 + m^2} \right) = 0
	\end{align*}

\end{example}

\end{document}




















