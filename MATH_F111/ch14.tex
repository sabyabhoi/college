\documentclass[12pt]{article}

\usepackage{amsmath}
\usepackage{amssymb}
\usepackage{amsthm}

\theoremstyle{definition}
\newtheorem*{example}{Example}
\theoremstyle{definition}
\newtheorem*{defn}{Definition}

\begin{document}

\title{Partial Derivatives}
\author{Sabyasachi Bhoi}
\date{Dec 18, 2020}
\maketitle

\begin{defn}
	Suppose $D$ is a set of n-tuples of real numbers $(x_1, x_2, \cdots x_n)$. A \textbf{real-valued function} $f$ on $D$ is a rule that assigns a unique (single) real number
	\begin{equation*}
		w = f(x_1, x_2, \cdots, x_n)
	\end{equation*}
to each element in $D$. The set $D$ is the function's domain. The set of w-values taken on by $f$ is the function's range. 
\end{defn}

\begin{defn}
	A point $(x_0, y_0)$ in a region R in the xy-plane is an interior point of R if it is the center of a disk of positive radius that lies entirely in $R$. A point $(x_0, y_0)$ is a boundary point of $R$ if every disk centered at $(x_0, y_0)$ contains points that lie outside of $R$ as well as points that lie in $R$.
\end{defn}

\begin{defn}
	A region in the plane is \textbf{bounded} if it lies inside a disk of finite radius. A region is \textbf{unbounded} if it is not bounded. 
\end{defn}

\begin{defn}
	The set of points in the plane where a function $f(x, y)$ has a constant value $f(x, y) = c$ is called the \textbf{level curve} of $f$. The set of all points $(x, y, f(x, y))$ in space, for $(x, y)$ in the domain of $f$, is called the \textbf{graph} of $f$. The graph of $f$ is also called the \textbf{surface} $z = f(x, y)$.  
\end{defn}

\end{document}




















