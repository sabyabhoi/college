\documentclass[12pt]{article}

\usepackage{amsmath}
\usepackage{amssymb}
\usepackage{amsthm}

\theoremstyle{definition}
\newtheorem*{example}{Example}

\theoremstyle{definition}
\newtheorem*{defn}{Definition}

\begin{document}
\title{Quantum Chemistry}
\author{Sabyasachi Bhoi}
\date{Dec 14, 2020}
\maketitle

\section{Vibrational Harmonics}

\subsubsection*{Harmonic Motion:} 

\[
	F = -kx, \quad \textrm{where $k$ is the force constant}
\]

\subsubsection*{Potential Energy:} 
\[
	V = \frac{1}{2} kx^2
\]

\subsubsection*{Schr\"odinger Equation}
\[
	- \frac{\hbar^2}{2m} \frac{d^2\Psi}{d x^2} + \frac{1}{2} kx^2\Psi = E\Psi
\]

\subsection{Energy Levels}

\[
	E_v = \left( v + \frac{1}{2} \right) h\nu
\]
Here, $\nu = \frac{1}{2\pi} \sqrt{\frac{k}{m}}$ and $v$ is the vibrational quantum number.

\section{Particle in a 2D ring}

\textbf{Convention:}
\begin{align*}
	\textrm{Clockwise rotation} \rightarrow m_l < 0\\
	\textrm{Anti-Clockwise rotation} \rightarrow m_l > 0\\
\end{align*}

\subsection{Energy level}

\begin{align*}
	&J_z = \pm \frac{hr}{\lambda}\\
	&\lambda = \frac{2\pi r}{m_l}\\
\end{align*}

\[
	\textrm{Angular Momentum}\; J_z = m_l \hbar 
\]

\[
	E_{m_l} = \frac{m_l^2\hbar^2}{2mr^2} = \frac{m_l^2\hbar^2}{2I}, \quad \textrm{where, $m_l$ is a quantum number.}  
\]

\section{Rotation in 3D - Rigid Rotor}
Classically, $E = \frac{J^2}{2mR^2} = \frac{J^2}{2I}$  

\subsection{Hamiltonian for motion in 3D}
\[
	\hat{H} = - \frac{\hbar^2}{2m} \nabla^2 + V
\]

Here, $\nabla^2$ is the Laplacian Operator (in spherical co-ordinate), which is defined as:

\[
	\nabla^2 = \frac{\partial^2}{\partial r^2} + \frac{1}{r} \frac{\partial}{\partial r} + \frac{1}{r^2} \Lambda^2 \quad \textrm{where, $\Lambda^2$ is the Legendrian Operator}
\]

\[
	\Lambda^2 = \frac{1}{sin^2\theta} \frac{\partial^2}{\partial\phi^2} + \frac{1}{sin\theta} \frac{\partial sin\theta}{\partial \theta} \frac{\partial}{\partial\theta}  
\]

\[
	- \frac{\hbar^2}{2m} \nabla^2 \Psi(\theta, \phi) = E \Psi(\theta, \phi)
\]
\section{Rigid rotor wavefunctions}
$\Psi(\theta, \phi)$ is separated into $\theta$ part and the $\phi$ part, that is, $\varPsi(\theta, \phi) = \Theta(\theta)\Phi(\phi)$.

\subsection{Energy levels for Rigid Rotor}
\[
	E_l = l(l + 1) \frac{\hbar^2}{2I} 
\]

\begin{align*}
	J = \hbar \sqrt{l(l+1)}
\end{align*}

\section{Energy levels}%
\label{cha:Energy levels}

\[
	E_n = \frac{ -hcRZ^2 }{n^2} \quad \textrm{where $R = 109677cm^{-1}$}
\]
Energy only depends on $n$ and not $l$ or $m_l$.

\[
	\frac{1 }{\lambda } = R \left( \frac{1 }{n_1^2} - \frac{1 }{n_2^2}  \right)
\]


\section{Ground state of Hydrogen Atom}%
\label{sec:Ground state of Hydrogen Atom}

\[
	n = 0,\;l = 0,\;m_l = 0 \rightarrow \textrm{1s orbital}
\]

\[
	\Psi_{n, l, m_l}(r, \theta, \phi) = R_{n, l}(r) \Theta_{l, m_l}(\theta) \Phi_{m_l}(\phi) = R_{n, l}(r) Y_{l, m_l}(\theta, \phi)
\]

\begin{example}
	\[
		\Psi_{100} = \left[ \left( \frac{4}{a_0^3} \right)^{1/2} e^{-r/a_0}  \right] \left[ \left( 4\pi \right)^{-1/2} \right]
	\]
\end{example}

No nodes in this case.

\section{Radial Wavefunctions}%
\label{sec:Radial Wavefunctions}

\[
	R_{n, l}(r) \propto r^l \times \textrm{Polynomial of degree $n-l-1$} \times e^{- \frac{Zr }{na_0} }
\]

\begin{itemize}
	\item The first factor determines the behaviour at r = 0. (only $l = 0$ wavefunctions are non-zero at the origin)
	\item The second factor determines the number of nodes at $n - l - 1$
	\item The third ensures that the function goes to 0 as $r \to \infty$
\end{itemize}

\begin{example}
	\[
		\Psi_{2s} \propto \left( 2 - \frac{Zr}{a_0}  \right)\;e^{ -\frac{Zr}{2a_0} }
	\]
\end{example}

\subsection{Probability Density}%
\label{sub:Probability Density}

\[
	\textrm{Volume of spherical shell} = 4 \pi r^2 dr
\]

\begin{align*}
	\textrm{Probability}\;P(r)dr &= \left| \Psi \right|^2 \times 4\pi r^2 dr\\
	\implies P(r) &= 4 \pi r^2 \Psi^2\quad \textrm{where, $P(r)$ is the Probability Density} 
\end{align*}

\section{Electron spin}

\begin{itemize}
	\item Intrinsic angular momentum, and associated magnetic momentum, characteristic of electron, a property like its mass, charge etc.
	\item Any spinning body has the ability to generate a magnetic field. 
	\item First indicated by spectroscopic features: eg. the doublet Na 'D-lines'
\end{itemize}

This was confirmed by \textbf{Stern-Gerlach Experiment}, from where we concluded that an electron has two spin states: $+\frac{1}{2}$ or $-\frac{1}{2}$.  

Electron described by intrinsic spin angular momentum quantum number $s = \frac{1}{2}$.

\[
	\textrm{magnitude of angular momentum} = \left[ \frac{1}{2} \left( \frac{1}{2} + 1 \right) \right] \hbar = \left( \frac{\sqrt{3}}{2}  \right) \hbar
\]

The component of electron spin along any axis in space, say the z-axis can take on one of the two values $\frac{\hbar}{2}$ or $-\frac{\hbar}{2}$.

\[
	m_s = \frac{1}{2} ( \textrm{$\alpha$ state} )\quad or \; -\frac{1}{2} ( \textrm{$\beta$ state} ) 
\]

\textbf{Convention:} 
\begin{align*}
	\textrm{clockwise rotation} &\rightarrow + \frac{1}{2} \\
	\textrm{ant-clockwise rotation} &\rightarrow + -\frac{1}{2} \\
\end{align*}

\subsection{Spin Orbital}
one electron wavefunction written as a product of the space part and the spin part.

\[
	\textrm{Spin Orbital} = \textrm{Spatial orbital} \times \textrm{Spin function} = 1 s \alpha
\]

\begin{itemize}
	\item All fundamental particles have characteristic spin values.
	\item Protons and neutrons are also spin $\frac{1}{2}$ particles like electrons.
	\item Photons are spin 1 particles.
\end{itemize}

All particles may be classified as \textbf{Bosons} ($s = 0, 1, 2, \dots$) or \textbf{Fermions} ($s = \frac{1}{2}, \frac{3}{2}, \frac{5}{2}, \dots$) 

\subsection{Spectral Transition}

In the one electron atom, spectral line occurs when an electron makes a transition between shells with principal quantum number $n_i$ and $n_f$. However, not all transitions between all available orbitals are possible or 'allowed'.\\

\textbf{Example:}  
\begin{itemize}
	\item 2p to 1s, and 3p to 1s are \textit{allowed} transitions.
	\item 2s to 1s or 3d to 1s are \textit{forbidden} transitions.
\end{itemize}

A statement about which spectroscopic transitions are allowed is called the \textit{selection rule} .

\subsection{Selection rule for one electron atom}

the selection rule here is just the conservation of angular momentum.

\textbf{Transitions:} Change in angular momentum of the electron must compensate for the angular momentum carried away by the photon. 

\begin{align*}
	\Delta l &=  0, \pm 1 \\
	\Delta m_l &=  0, \pm 1 \\
\end{align*}

\section{Multi-Electron Atoms}

\textbf{Orbital Approximation:} Approximate the wavefunction as a product of one electron functions or orbitals.

\[
	\Psi(1, 2, \dots) = \Psi_1(1)\Psi_2(2) \dots
\]
Each orbital may be thought of as being hydrogen like with an effective nuclear charge.

\begin{example} Helium Atom (2 electrons)
	\[
		\hat{H} = - \frac{\hbar^2}{2m} \nabla^2_1 - \frac{Ze^2}{4 \pi \varepsilon_0r_1} - \frac{\hbar^2}{2m}\nabla^2_2 - \frac{Ze^2}{4\pi\varepsilon_0 r_2} + \frac{e^2}{4\pi\varepsilon_0 r_{12}} 
	\]
	\[
		\hat{H} \approx - \frac{\hbar^2}{2m} \nabla^2_1 - \frac{Z}{4\pi\varepsilon_0 r_1} - \frac{\hbar^2}{2m}\nabla^2_2 - \frac{Z}{4\pi\varepsilon_0 r_2} = \hat{H_1} + \hat{H_2}
	\]

\end{example}

\subsection{The Pauli Principle}

\subsubsection{Pauli Exclusion Principle}
\begin{defn}
	No more than two electrons may occupy any given orbital, and if two do occupy one orbital, then their spins must be paired.
\end{defn}

\subsubsection{The Pauli Principle}
\begin{defn}
	When the labels of any two identical fermions are exchanged, the total wavefunction changes sign; when the labels of any two identical bosons are exchanged, the total wavefunction retains the same sign.
\end{defn}

Therefore, in case of \textit{fermions}, 
\[
	\Psi(2, 1) = -\Psi(1, 2)
\]

But practically speaking, we make the following approximation:
\[
	\Psi(1, 2) = \Psi(1) \Psi(2)
\]

This approximation would fail by Pauli's Principle. Therefore, we add a spin part to this to make it of the form
\[
	\Psi(1, 2) = \Psi(1) \Psi(2) \left\{ \textrm{Spin Part} \right\}
\]

Possibility of spin configurations:
\begin{enumerate}
	\item \textbf{both up-spin:} $\alpha(1)\alpha(2)$
	\item \textbf{both down-spin:} $\beta(1)\beta(2)$
	\item \textbf{first up-spin and second down-spin:} $\alpha(1)\beta(2)$
	\item \textbf{first down-spin and second up-spin:} $\beta(1)\alpha(2)$
\end{enumerate}

\subsubsection{Form of antisymmetric wavefunction}

Possible spin wavefunctions:
\begin{align}
	& \alpha(1)\alpha(2) \\
	& \beta(1)\beta(2) \\
	& \alpha(1)\beta(2) \\
	& \beta(1)\alpha(2)
\end{align}

Since all of them are equally likely, we don't randomly select a single wavefunction; instead we take a linear combination of the two.
\begin{align*}
	\sigma_+(1, 2) &=  2^{-1/2} \Psi(1)\Psi(2) \left\{ \alpha(1)\beta(2) + \beta(1)\alpha(2) \right\} \\
	\sigma_-(1, 2) &=  2^{-1/2} \Psi(1)\Psi(2) \left\{ \alpha(1)\beta(2) - \beta(1)\alpha(2) \right\}
\end{align*}

We can observe that the second equation is anti-symmetric, since 
\[
	\sigma_-(1, 2) = - \sigma_-(2, 1)
\]

From these calculations, we can get 4 different wavefunctions:
\begin{align*}
	&\Psi(1)\Psi(2)\alpha(1)\alpha(2) \\
	&\Psi(1)\Psi(2)\beta(1)\beta(2) \\
	&\Psi(1)\Psi(2) \sigma_+(1, 2) \\
	&\Psi(1)\Psi(2) \sigma_-(1, 2)
\end{align*}

\subsection{Anti-symmetry in the systems with upaired electrons}

\end{document}























