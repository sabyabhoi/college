\documentclass[12pt]{article}

\usepackage{amsmath}
\usepackage{amssymb}
\usepackage{amsthm}
\usepackage{modiagram}
\usepackage{chemfig}

\theoremstyle{definition}
\newtheorem*{example}{Example}

\begin{document}
\title{Atomic Bonding}
\author{Sabyasachi Bhoi}
\date{Dec 15, 2020}
\maketitle

\section{VB Theory: For Polyatomic molecules}

\subsection{Two modifications}
\begin{itemize}
	\item Promotion of electron
	\item Hybridization
\end{itemize}

\subsection{Hybridized Orbitals}
Consider $n$ atomic orbitals with wave functions $\varphi_1 \dots \varphi_n$ mix up.

\begin{align*}
	\Psi_1 &= \varphi_{2s} + \varphi_{2p_x} + \varphi_{2p_y} + \varphi_{2p_z} \\
	\Psi_2 &= \varphi_{2s} + \varphi_{2p_x} - \varphi_{2p_y} - \varphi_{2p_z} \\
	\Psi_3 &= \varphi_{2s} - \varphi_{2p_x} - \varphi_{2p_y} + \varphi_{2p_z} \\
	\Psi_4 &= \varphi_{2s} - \varphi_{2p_x} + \varphi_{2p_y} - \varphi_{2p_z} \\
\end{align*}

Electron in each of these $sp^3$ hybrid orbitals pairs up with electrons in H $1s$ orbital forming 4 identical $\sigma$ bonds.

\section{Superposition of wavefunctions}

\begin{equation*}
	\int{{\Psi_1 \Psi_2}}\: d{\tau} = 0
\end{equation*}

\begin{align*}
	\Psi_1 &= \varphi_{2s} + \varphi_{2p_z} \\
	\Psi_1 &= \varphi_{2s} - \varphi_{2p_z} \\
\end{align*}

\begin{equation*}
	\int{{\Psi_1 \Psi_2}}\: d{\tau} = \int {\varphi_{2s}^2p} \: d{\tau} - \int{\varphi_{2p_z}^2} \: d{\tau}
\end{equation*}

\section{Molecular Orbital Theory}
Here, electrons do not belong to particular bonds, but spread \textbf{throughout} the entire molecule.

The wavefunction of hydrogen atom A and hydrogen atom B can interact either constructively or destructively.

\begin{equation*}
	\Psi_{\pm} = N(A \pm B)
\end{equation*}

Assume that molecular orbitals are linear combinations of atomic orbitals:
\begin{equation*}
	\Psi = \sum^{n}_{i=1} C_i\varphi_i \quad \textrm{where, $C_i$ is the mixing coefficient}
\end{equation*}

\begin{equation*}
	\hat{H_e} = \check{T_e} - \frac{e^2 }{4 \pi \epsilon_0 r_a} - \frac{e^2}{4 \pi \epsilon_0 r_b} + \frac{e^2}{4 \pi \epsilon_0 R} 
\end{equation*}

\begin{equation*}
	\Psi_1 = N_1(\varphi_a + \varphi_b)
	\Psi_2 = N_1(\varphi_a - \varphi_b)
\end{equation*}

\subsection{Molecular Orbital wave function($H_2^+$ molecule)}

Constructive interference:
\begin{equation*}
	\Psi_\sigma\;or\; \Psi_+ = \frac{1}{\sqrt{2}} 
\end{equation*}

\begin{equation*}
	\Psi_+^2 = N^2 \left( A^2 + B^2 + 2AB \right)
\end{equation*}

\begin{itemize}
	\item Predicts an equilibrium bond length of 130 pm and bond dissociation energy of $171 kJ/mole$
	\item Experimental values are 106 pm and $250 kJ/mole$
	\item Reasonable agreement given crude nature of description
\end{itemize}

\subsection{Antibonding Molecular Orbital}

Energy decreases monotonically as a function of the inter-nuclear separation with no minimum

\begin{equation*}
	\Psi_-^2 = N^2 \left( A^2 + B^2 - 2AB \right)
\end{equation*}

\begin{equation*}
	- \frac{\hbar^2}{2m} \frac{\partial^2\Psi}{\partial x^2} = E\Psi
\end{equation*}

\subsection{Energy levels in Molecular orbitals for $H_2^+$ molecule}

Number of nodal planes = $0 (\sigma), \; 1(\pi)$

\end{document}
